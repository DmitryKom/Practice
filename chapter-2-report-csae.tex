\chapter{\label{ch:ch02}ГЛАВА 2. Практическая часть}
\section{\label{sec:ch02/sec01}Техническое задание}
Создаётся окно в котором имеется менюбар с кнопками <<Menu>> и <<About>>. В кнопке <<Menu>>: <<В меню>>, <<Выбор режима>>, <<Выйти>>, а в кнопке <<About>> информация о создателе. В самом окне при создании появляются кнопки: <<Выбор режима>>, <<Выйти>>. Кнопка <<Выбор режима>> предлагает 4 режима игры: <<игра с ботом с готовой расстановкой>>, <<игра с ботом с расстановкой>>, <<игра 1 на 1 с человеком>> и <<расширенный режим игры 1 на 1>>. 
Описание каждого режима:Ф
\begin{enumerate}
\item <<Игра с ботом с готовой расстановкой>> - стандартный размер поля 10 x 10 клеток, корабли бота расставлены заранее, корабли игрока необходимо расставить самостоятельно.
\item <<Игра с ботом с расстановкой>> - стандартный размер поля 10 x 10 клеток, корабли бота расставляются в случайном порядке, корабли игрока необходимо расставить самостоятельно.
\item <<Игра 1 на 1 с человеком>> - стандартный размер поля 10 x 10 клеток, игроки в порядке очереди самостоятельно расставляют корабли.
\item <<Расширенный режим игры 1 на 1>> - необходимо выбрать размер поля \- n × n клеток (для двух игроков поля одинаковые), задать общее количество <<мин>> (задается число кратное 2, так как <<мины>> разделяются на два поля) и выбрать количество игровых кораблей. <<Мины>> расставляются по карте случайным образом. При попадании в <<мину>>, взрывается клетка, где находилась мина, а так же клетка над миной и слева от  нее.  Игроки в порядке очереди самостоятельно расставляют корабли.
\end{enumerate}
