\chapter{\label{ch:ch01}Теоритическая глава} % Нужно сделать главу в содержании заглавными буквами
\section{\label{sec:ch01/sec01}Общие сведения об игре <<Морской бой>>}
\textbf{Правила размещения кораблей (флота)}~\cite{p5}

Игровое поле — обычно квадрат 10×10 у каждого игрока, на котором размещается флот кораблей. Горизонтали обычно нумеруются сверху вниз, а вертикали помечаются буквами слева направо. При этом используются буквы русского алфавита от «а» до «к» (буквы «ё» и «й» обычно пропускаются) либо от «а» до «и» (с использованием буквы «ё»), либо буквы латинского алфавита от «a» до «j». Иногда используется слово «республика» или «снегурочка», так как в этих 10-буквенных словах ни одна буква не повторяется. Поскольку существуют различные варианты задания системы координат, то об этом лучше заранее договориться.

Размещаются:
\begin{itemize}
\item 1 корабль — ряд из 4 клеток («четырёхпалубный»; линкор);
\item 2 корабля — ряд из 3 клеток («трёхпалубные»; крейсера);
\item 3 корабля — ряд из 2 клеток («двухпалубные»; эсминцы);
\item 4 корабля — 1 клетка («однопалубные»; торпедные катера).
\end{itemize}

При размещении корабли не могут касаться друг друга сторонами и углами. Встречаются, однако, варианты, когда касание углами не запрещается. Встречаются также варианты игры, когда корабли могут размещаться буквой «Г» («трёх-» и «четырёхпалубные»), квадратом или зигзагом («четырёхпалубные»). Кроме того, есть варианты с другим набором кораблей (напр., один пятипалубный, два четырёхпалубных и т. д.) и/или другой формой поля (15×15 для пятипалубных).

Рядом со «своим» полем чертится «чужое» такого же размера, только пустое. Это участок моря, где плавают корабли противника.

При попадании в корабль противника — на чужом поле ставится крестик, при холостом выстреле — точка. Попавший стреляет ещё раз.

Самыми уязвимыми являются линкор и торпедный катер: первый из-за крупных размеров, в связи с чем его сравнительно легко найти, а второй из-за того, что топится с одного удара, хотя его найти достаточно сложно.

\textbf{Потопление кораблей противника}~\cite{p13}

Игрок, выполняющий ход, совершает выстрел — называет координаты клетки, в которой, по его мнению, находится корабль противника, например, «В1».
\begin{itemize}
\item Если выстрел пришёлся в клетку, не занятую ни одним кораблём противника, то следует ответ «Мимо!» и стрелявший игрок ставит на чужом квадрате в этом месте точку. Право хода переходит к сопернику.
\item Если выстрел пришёлся в клетку, где находится многопалубный корабль (размером больше чем 1 клетка), то следует ответ «Ранил(а)!» или «Попал(а)!». Стрелявший игрок ставит на чужом поле в эту клетку крестик, а его противник ставит крестик на своём поле также в эту клетку. Стрелявший игрок получает право на ещё один выстрел.
\item Если выстрел пришёлся в клетку, где находится однопалубный корабль, или последнюю непоражённую клетку многопалубного корабля, то следует ответ «Убил(а)!» или «Потопил(а)!». Оба игрока отмечают потопленный корабль на листе. Стрелявший игрок получает право на ещё один выстрел.
\end{itemize}
Победителем считается тот, кто первым потопит все 10 кораблей противника.

\textbf{Оптимальная стратегия}~\cite{p6}

В игре морской бой всегда есть элемент случайности, но его можно свести к минимуму. Прежде чем переходить непосредственно к поиску оптимальной стратегии, необходимо озвучить одну очевидную вещь: вероятность попасть по кораблю противника тем выше, чем меньше непроверенных клеток осталось на его поле, аналогично вероятность попадания по вашим кораблям тем ниже, чем больше непровереных клеток осталось на вашем поле. Т.о. для эффективной игры нужно научиться сразу двум вещам: оптимальной стрельбе по противнику и оптимальному своих размещению кораблей.

\section{\label{sec:ch01/sec02}Основные характеристики языка Python}

Python~\cite{p2} представляет популярный высокоуровневый язык программирования, который предназначен для создания приложений различных типов. Это и веб-приложения~\cite{p9}, и игры~\cite{p8}, и настольные программы, и работа с базами данных. Довольно большое распространение Python получил в области машинного обучения и исследований искусственного интеллекта. Синтаксис Python лаконичен и понятен, что упрощает разработку и понимание кода. Это позволяет программистам быстрее создавать игровые приложения и легче поддерживать их.

Python является бесплатным и открытым исходным кодом, что делает его доступным для всех-\cite{p10}. 

Также Python можно использовать для работы с кроссплатворменными приложениями. Это означает, что приложения, разработанные на Python, могут быть запущены на различных операционных системах без необходимости внесения изменений в исходный код.

Интерпретатор Python легко расширяется за счет новых функций и типов данных, реализованных на C или C ++ (или других языках, вызываемых из C). Python также подходит в качестве языка расширения для настраиваемых приложений.

Недостатками языка~\cite{p14} являются зачастую более низкая скорость работы и более высокое потребление памяти написанных на нём программ по сравнению с аналогичным кодом, написанным на компилируемых языках, таких как C или C++.

\section{\label{sec:ch01/sec03}Кроссплатформенность}
Кроссплатформенность ~\cite{p12} --- способность программного обеспечения работать с несколькими аппаратными платформами или операционными системами. Обеспечивается благодаря использованию высокоуровневых языков программирования, сред разработки и выполнения, поддерживающих условную компиляцию, компоновку и выполнение кода для различных платформ. Типичным примером является программное обеспечение, предназначенное для работы в операционных системах Linux и Windows одновременно.

Python поддерживает разработку кроссплатформенных приложений благодаря своей гибкости и множеству доступных библиотек и инструментов~\cite{p3}.

\section{\label{sec:ch01/sec04}Библиотека wxPython}

WxPython~\cite{p1} - это кроссплатформенная библиотека, основанная на wxWidgets, которая предназначена для языка программирования C++. WxPython используется для разработки графических пользовательских интерфейсов (GUI)~\cite{p4} на языке программирования Python. Она предоставляет набор инструментов и компонентов для создания разнообразных приложений с графическим интерфейсом, таких как оконные приложения, инструменты редактирования, игры~\cite{p11} и другие.

Описание основных характеристик и возможностей библиотеки wxPython:
\begin{itemize}
\item Кроссплатформенность: wxPython позволяет разрабатывать приложения, которые могут быть запущены на различных операционных системах, включая Windows, macOS и Linux.
\item Широкий набор компонентов интерфейса: библиотека предоставляет множество готовых компонентов интерфейса, таких как кнопки, поля ввода, таблицы, меню и многое другое. Это позволяет быстро создавать разнообразные элементы пользовательского интерфейса.
\item Гибкость и настраиваемость: wxPython предоставляет разработчикам широкие возможности для настройки внешнего вида и поведения компонентов интерфейса. Он поддерживает различные стили, темы оформления и возможности расширения функциональности с помощью собственных компонентов.
\item Обширная документация и активное сообщество: wxPython обладает обширной документацией, примерами кода и учебными материалами, что облегчает изучение и использование библиотеки. Кроме того, есть активное сообщество разработчиков, которые готовы помочь и поддержать друг друга.
\end{itemize}

wxPython предоставляет широкий спектр модулей~\cite{p15} и классов для создания различных элементов графического пользовательского интерфейса. Вот краткое описание некоторых основных модулей библиотеки:
\begin{itemize}
\item \textbf{wx}: базовый модуль wxPython. Классы в этом модуле являются наиболее часто используемыми классами для wxPython;
\item \textbf{wx.grid}: Grid и связанные классы в этом модуле предоставляют функциональность, аналогичную электронной таблице, где приложение может отображать строки и столбцы данных различных типов, которые пользователь может редактировать и иным образом взаимодействовать с ними;
\item \textbf{wx.aui}: предоставляет набор классов для реализации <<расширенного пользовательского интерфейса>>. Более конкретно, эти классы позволяют вам представить некоторые части вашего приложения в виде плавающих или закрепляемых панелей, записных книжек с плавающими вкладками и т.д.;
\item \textbf{wx.adv}: модуль для создания расширенных элементов управления GUI, таких как календари, списки выбора и т.д.
\end{itemize}
