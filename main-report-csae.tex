\documentclass[14pt, oneside]{altsu-report}

\worktype{Отчёт по технологической (проектно-технологической) практике на тему:}
\title{Разработка игры <<Морской бой>> на Python}
\author{Д.\,А.~Комягин}
\groupnumber{5.205-2}
\GradebookNumber{1337}
\supervisor{И.\,А.~Шмаков}
\supervisordegree{ст. преп. каф. ВТ и Э}
\ministry{Министерство науки и высшего образования}
\country{Российской Федерации}
\fulluniversityname{ФГБОУ ВО Алтайский государственный университет}
\institute{Институт цифровых технологий, электроники и физики}
\department{Кафедра вычислительной техники и электроники}
\shortdepartment{ВТиЭ}
\abstractRU{Ключевые слова: морской бой, игра, Python, wxPython, функция.}

\keysRU{На данный момент игровая индустрия имеет большую популярность и стремительно развивается. Разработка компьютерных игр является сложным и многогранным процессом, включающим в себя различные аспекты программирования, дизайна и тестирования.}


\date{\the\year}

% Подключение файлов с библиотекой.
\addbibresource{graduate-students.bib}

% Пакет для отладки отступов.
%\usepackage{showframe}

\begin{document}
\maketitle

\setcounter{page}{2}
\makeabstract
\tableofcontents

\chapter*{Введение}
\phantomsection\addcontentsline{toc}{chapter}{ВВЕДЕНИЕ}

\textbf{Актуальность:}
\begin{itemize}
\item популяризация программирования: изучение Python и wxPython способствует развитию навыков программирования, логического мышления и алгоритмизации;
\item Создание функционального приложения: разработанная игра будет являться функциональным приложением, которое можно использовать для развлечения или обучения;
\item Овладение навыками создания GUI: использование библиотеки wxPython в процессе разработки игры дает возможность создавать графические интерфейсы пользователя, что является ценным навыком для любого программиста;
\item Развитие игровой индустрии: создание игр – востребованное направление в IT-сфере;
\item Доступность инструментов: Python и wxPython – бесплатные и легкодоступные инструменты для разработки игр.
\end{itemize}

\textbf{Цель}

Разработать функциональную и интерактивную игру <<Морской бой>> на Python с использованием библиотеки wxPython, используя при этом:
\begin{itemize}
\item знания и навыки программирования на Python;
\item методы разработки графического интерфейса пользователя;
\item алгоритмы игровой логики.
\end{itemize}

\textbf{Задачи:}

\begin{itemize}
\item изучить различные варианты реализации игры <<Морской бой>>;
\item создать удобный и интуитивно понятный интерфейс для игры, используя элементы управления wxPython;
\item создать алгоритм для игры компьютера;
\item реализовать несколько режимов игры;
\item реализовать алгоритмы игрового процесса, включая:
\begin{itemize}
    \item размещение кораблей;
    \item выстрелы;
    \item определение победителя.
\end{itemize}
\end{itemize}

\textbf{Практическая значимость:}
\begin{itemize}
\item развитие навыков: работа над проектом позволит развить навыки программирования, проектирования интерфейсов, тестирования и отладки;
\item создание готового продукта: результатом работы станет функциональная игра, которую можно использовать для развлечения или в учебных целях;
\item возможность применения: полученные навыки могут быть применены для разработки других игр или программных решений.
\end{itemize}

% Подключение основной части (теория):
\chapter{\label{ch:ch01}ГЛАВА 1} % Нужно сделать главу в содержании заглавными буквами

% Подключение второй главы (практическая часть):
\chapter{\label{ch:ch02}ГЛАВА 2}


\chapter*{Заключение}
\addcontentsline{toc}{chapter}{Заключение}
В ходе выполнения курсовой работы была разработана игра <<Морской бой>> на языке программирования Python с использованием библиотеки wxPython. Разработка игры позволила мне освоить новые инструменты и библиотеки Python, а также улучшить навыки работы с переменными, циклами, условными операторами и функциями. Я убедился, что Python является отличным выбором для создания игр благодаря своему простому синтаксису и обширным библиотекам. Проект продемонстрировал возможности wxPython для создания графического интерфейса, а также предоставил возможность познакомиться с различными аспектами разработки игр, включая логику игры и искусственный интеллект.

В целом, разработка игры <<Морской бой>> представляла собой захватывающее и практическое путешествие в мир разработки игр. Этот проект помог изучить основные принципы создания игр, такие как работа с игровым полем, позиционирование объектов, обработка событий и другие. Учитывая стремительное развитие игровой индустрии, полученный опыт станет прочной основой для дальнейшего изучения и покорения вершин в этой сфере.
\newpage
\phantomsection\addcontentsline{toc}{chapter}{Список использованной литературы}
\printbibliography[title={Список использованной литературы}]
\nocite{*}

\appendix
\newpage
\chapter*{\raggedleft\label{appendix1}Приложение}
\phantomsection\addcontentsline{toc}{chapter}{ПРИЛОЖЕНИЕ}
%\section*{\centering\label{code:appendix}Текст программы}

\begin{center}
\label{code:appendix}Текст программы
\end{center}
\begin{code}
\captionof*{listing}{\centering\label{code:pi-example}Код создания и запуска основного окна игры}
\vspace{-1cm}\inputminted{Python}{code/sea_battle_panel.py}
\end{code}
\begin{code}
\captionof*{listing}{\centering\label{code:pi-example}Код режима с готовой расстановкой компьютера}
\vspace{-1cm}\inputminted{Python}{code/statik_bot.py}
\end{code}
\begin{code}
\captionof*{listing}{\centering\label{code:pi-example}Код режима с генерацией расставновки компьютера}
\vspace{-1cm}\inputminted{Python}{code/random_bot.py}
\end{code}
\begin{code}
\captionof*{listing}{\centering\label{code:pi-example}Код режима для игры с другом}
\vspace{-1cm}\inputminted{Python}{code/1x1.py}
\end{code}
\begin{code}
\captionof*{listing}{\centering\label{code:pi-example}Код расширенного режима}
\vspace{-1cm}\inputminted{Python}{code/more_game.py}
\end{code}
\end{document}

