\documentclass[14pt, oneside]{altsu-report}

\worktype{Отчёт по технологической (проектно-технологической) практике на тему:}
\title{Разработка игры <<Морской бой>> на Python}
\author{Д.\,А.~Комягин}
\groupnumber{5.205-2}
\GradebookNumber{1337}
\supervisor{И.\,А.~Шмаков}
\supervisordegree{ст. преп. каф. ВТ и Э}
\ministry{Министерство науки и высшего образования}
\country{Российской Федерации}
\fulluniversityname{ФГБОУ ВО Алтайский государственный университет}
\institute{Институт цифровых технологий, электроники и физики}
\department{Кафедра вычислительной техники и электроники}
\shortdepartment{ВТиЭ}
\abstractRU{Ключевые слова: морской бой, игра, Python, wxPython, функция.}

\keysRU{На данный игровая индустрия имеет большую популярность и стремительно развивается. Разработка компьютерных игр является сложным и многогранным процессом, включающим в себя различные аспекты программирования, дизайна и тестирования.}


\date{\the\year}

% Подключение файлов с библиотекой.
\addbibresource{graduate-students.bib}

% Пакет для отладки отступов.
%\usepackage{showframe}

\begin{document}
\maketitle

\setcounter{page}{2}
\makeabstract
\tableofcontents

\chapter*{Введение}
\phantomsection\addcontentsline{toc}{chapter}{ВВЕДЕНИЕ}

\textbf{Актуальность:}
\begin{itemize}
\item популяризация программирования: изучение Python и wxPython способствует развитию навыков программирования, логического мышления и алгоритмизации;
\item Создание функционального приложения: разработанная игра будет являться функциональным приложением, которое можно использовать для развлечения или обучения;
\item Овладение навыками создания GUI: использование библиотеки wxPython в процессе разработки игры дает возможность создавать графические интерфейсы пользователя, что является ценным навыком для любого программиста;
\item развитие игровой индустрии: создание игр – востребованное направление в IT-сфере;
\item доступность инструментов: Python и wxPython – бесплатные и легкодоступные инструменты для разработки игр.
\end{itemize}

\textbf{Цель}

Разработать функциональную и интерактивную игру <<Морской бой>> на Python с использованием библиотеки wxPython, используя при этом:
\begin{itemize}
\item знания и навыки программирования на Python;
\item методы разработки графического интерфейса пользователя;
\item алгоритмы игровой логики.
\end{itemize}

\textbf{Задачи:}

\begin{itemize}
\item изучить различные варианты реализации игры <<Морской бой>>;
\item создать удобный и интуитивно понятный интерфейс для игры, используя элементы управления wxPython;
\item создать алгоритм для игры компьютера;
\item реализовать несколько режимов игры;
\item реализовать алгоритмы игрового процесса, включая:
\begin{itemize}
    \item размещение кораблей;
    \item выстрелы;
    \item определение победителя.
\end{itemize}
\end{itemize}

\textbf{Практическая значимость:}
\begin{itemize}
\item развитие навыков: работа над проектом позволит развить навыки программирования, проектирования интерфейсов, тестирования и отладки;
\item создание готового продукта: результатом работы станет функциональная игра, которую можно использовать для развлечения или в учебных целях;
\item возможность применения: полученные навыки могут быть применены для разработки других игр или программных решений.
\end{itemize}

% Подключение основной части (теория):
\chapter{\label{ch:ch01}Теоритическая глава} % Нужно сделать главу в содержании заглавными буквами
\section{\label{sec:ch01/sec01}Общие сведения об игре <<Морской бой>>}
\textbf{Правила размещения кораблей (флота)}

Игровое поле — обычно квадрат 10×10 у каждого игрока, на котором размещается флот кораблей. Горизонтали обычно нумеруются сверху вниз, а вертикали помечаются буквами слева направо. При этом используются буквы русского алфавита от «а» до «к» (буквы «ё» и «й» обычно пропускаются) либо от «а» до «и» (с использованием буквы «ё»), либо буквы латинского алфавита от «a» до «j». Иногда используется слово «республика» или «снегурочка», так как в этих 10-буквенных словах ни одна буква не повторяется. Поскольку существуют различные варианты задания системы координат, то об этом лучше заранее договориться.

Размещаются:
\begin{itemize}
\item 1 корабль — ряд из 4 клеток («четырёхпалубный»; линкор);
\item 2 корабля — ряд из 3 клеток («трёхпалубные»; крейсера);
\item 3 корабля — ряд из 2 клеток («двухпалубные»; эсминцы);
\item 4 корабля — 1 клетка («однопалубные»; торпедные катера).
\end{itemize}

При размещении корабли не могут касаться друг друга сторонами и углами. Встречаются, однако, варианты, когда касание углами не запрещается. Встречаются также варианты игры, когда корабли могут размещаться буквой «Г» («трёх-» и «четырёхпалубные»), квадратом или зигзагом («четырёхпалубные»). Кроме того, есть варианты с другим набором кораблей (напр., один пятипалубный, два четырёхпалубных и т. д.) и/или другой формой поля (15×15 для пятипалубных).

Рядом со «своим» полем чертится «чужое» такого же размера, только пустое. Это участок моря, где плавают корабли противника.

При попадании в корабль противника — на чужом поле ставится крестик, при холостом выстреле — точка. Попавший стреляет ещё раз.

Самыми уязвимыми являются линкор и торпедный катер: первый из-за крупных размеров, в связи с чем его сравнительно легко найти, а второй из-за того, что топится с одного удара, хотя его найти достаточно сложно.

\textbf{Правила размещения кораблей (флота)} Потопление кораблей противника

Игрок, выполняющий ход, совершает выстрел — называет координаты клетки, в которой, по его мнению, находится корабль противника, например, «В1».
\begin{itemize}
\item Если выстрел пришёлся в клетку, не занятую ни одним кораблём противника, то следует ответ «Мимо!» и стрелявший игрок ставит на чужом квадрате в этом месте точку. Право хода переходит к сопернику.
\item Если выстрел пришёлся в клетку, где находится многопалубный корабль (размером больше чем 1 клетка), то следует ответ «Ранил(а)!» или «Попал(а)!». Стрелявший игрок ставит на чужом поле в эту клетку крестик, а его противник ставит крестик на своём поле также в эту клетку. Стрелявший игрок получает право на ещё один выстрел.
\item Если выстрел пришёлся в клетку, где находится однопалубный корабль, или последнюю непоражённую клетку многопалубного корабля, то следует ответ «Убил(а)!» или «Потопил(а)!». Оба игрока отмечают потопленный корабль на листе. Стрелявший игрок получает право на ещё один выстрел.
\end{itemize}
Победителем считается тот, кто первым потопит все 10 кораблей противника.

\section{\label{sec:ch01/sec02}Основные характеристики языка Python}

Руководство по Python
Python - простой в освоении язык программирования. Он обладает эффективными высокоуровневыми структурами данных и простым, но эффективным подходом к объектно-ориентированному программированию. Синтаксис Python лаконичен и понятен, что упрощает разработку и понимание кода. Это позволяет программистам быстрее создавать игровые приложения и легче поддерживать их.

Python является бесплатным и открытым исходным кодом, что делает его доступным для всех. 

Также Python можно использовать для работы с кроссплатворменными приложениями. Это означает, что приложения, разработанные на Python, могут быть запущены на различных операционных системах без необходимости внесения изменений в исходный код.

Интерпретатор Python легко расширяется за счет новых функций и типов данных, реализованных на C или C ++ (или других языках, вызываемых из C). Python также подходит в качестве языка расширения для настраиваемых приложений.

Недостатками языка являются зачастую более низкая скорость работы и более высокое потребление памяти написанных на нём программ по сравнению с аналогичным кодом, написанным на компилируемых языках, таких как C или C++.

\section{\label{sec:ch01/sec03}Кроссплатформенные приложения}

Кроссплатформенное приложение — это программное обеспечение, которое работает на нескольких платформах, таких как Windows, macOS, Linux, Android и iOS. Создание кроссплатформенных приложений позволяет сократить время и стоимость разработки, а также упрощает поддержку и обновление продукта.

Python поддерживает разработку кроссплатформенных приложений благодаря своей гибкости и множеству доступных библиотек и инструментов.

\section{\label{sec:ch01/sec04}Библиотека wxPython}

WxPython - это кроссплатформенная библиотека, основанная на wxWidgets, которая предназначена для языка программирования C++. WxPython используется для разработки графических пользовательских интерфейсов (GUI) на языке программирования Python. Она предоставляет набор инструментов и компонентов для создания разнообразных приложений с графическим интерфейсом, таких как оконные приложения, инструменты редактирования, игры и другие.

Описание основных характеристик и возможностей библиотеки wxPython:
\begin{itemize}
\item Кроссплатформенность: wxPython позволяет разрабатывать приложения, которые могут быть запущены на различных операционных системах, включая Windows, macOS и Linux.
\item Широкий набор компонентов интерфейса: Библиотека предоставляет множество готовых компонентов интерфейса, таких как кнопки, поля ввода, таблицы, меню и многое другое. Это позволяет быстро создавать разнообразные элементы пользовательского интерфейса.
\item Гибкость и настраиваемость: wxPython предоставляет разработчикам широкие возможности для настройки внешнего вида и поведения компонентов интерфейса. Он поддерживает различные стили, темы оформления и возможности расширения функциональности с помощью собственных компонентов.
\item Обширная документация и активное сообщество: wxPython обладает обширной документацией, примерами кода и учебными материалами, что облегчает изучение и использование библиотеки. Кроме того, есть активное сообщество разработчиков, которые готовы помочь и поддержать друг друга.
\end{itemize}

wxPython предоставляет широкий спектр модулей и классов для создания различных элементов графического пользовательского интерфейса. Вот краткое описание некоторых основных модулей библиотеки:
\begin{itemize}
\item \textbf{wx}: базовый модуль wxPython. Классы в этом модуле являются наиболее часто используемыми классами для wxPython;
\item \textbf{wx.grid}: Grid и связанные классы в этом модуле предоставляют функциональность, аналогичную электронной таблице, где приложение может отображать строки и столбцы данных различных типов, которые пользователь может редактировать и иным образом взаимодействовать с ними;
\item \textbf{wx.aui}: предоставляет набор классов для реализации <<расширенного пользовательского интерфейса>>. Более конкретно, эти классы позволяют вам представить некоторые части вашего приложения в виде плавающих или закрепляемых панелей, записных книжек с плавающими вкладками и т.д.;
\item \textbf{wx.adv}: модуль для создания расширенных элементов управления GUI, таких как календари, списки выбора и т.д.
\end{itemize}

% Подключение второй главы (практическая часть):
\chapter{\label{ch:ch02}ГЛАВА 2. Практическая часть}
\section{\label{sec:ch02/sec01}Техническое задание}
Создаётся окно в котором имеется менюбар с кнопками <<Menu>> и <<About>>. В кнопке <<Menu>>: <<В меню>>, <<Выбор режима>>, <<Выйти>>, а в кнопке <<About>> информация о создателе. В самом окне при создании появляются кнопки: <<Выбор режима>>, <<Выйти>>. Кнопка <<Выбор режима>> предлагает 4 режима игры: <<игра с ботом с готовой расстановкой>>, <<игра с ботом с расстановкой>>, <<игра 1 на 1 с человеком>> и <<расширенный режим игры 1 на 1>>. 
Описание каждого режима:Ф
\begin{enumerate}
\item <<Игра с ботом с готовой расстановкой>> - стандартный размер поля 10 x 10 клеток, корабли бота расставлены заранее, корабли игрока необходимо расставить самостоятельно.
\item <<Игра с ботом с расстановкой>> - стандартный размер поля 10 x 10 клеток, корабли бота расставляются в случайном порядке, корабли игрока необходимо расставить самостоятельно.
\item <<Игра 1 на 1 с человеком>> - стандартный размер поля 10 x 10 клеток, игроки в порядке очереди самостоятельно расставляют корабли.
\item <<Расширенный режим игры 1 на 1>> - необходимо выбрать размер поля \- n × n клеток (для двух игроков поля одинаковые), задать общее количество <<мин>> (задается число кратное 2, так как <<мины>> разделяются на два поля) и выбрать количество игровых кораблей. <<Мины>> расставляются по карте случайным образом. При попадании в <<мину>>, взрывается клетка, где находилась мина, а так же клетка над миной и слева от  нее.  Игроки в порядке очереди самостоятельно расставляют корабли.
\end{enumerate}


\chapter*{Заключение}
\addcontentsline{toc}{chapter}{Заключение}
В ходе выполнения курсовой работы была разработана игра <<Морской бой>> на Python с использованием библиотеки wxPython. Проект продемонстрировал возможности wxPython для создания графического интерфейса и интеграции его с логикой игры. В дальнейшем можно расширить функциональность игры, улучшив графику и интерфейс пользователя, а также оработав искусственный интеллект для игры с компьютером.

\newpage
\phantomsection\addcontentsline{toc}{chapter}{Список использованной литературы}
\printbibliography[title={Список использованной литературы}]
\nocite{*}

\appendix
\newpage
\chapter*{\raggedleft\label{appendix1}Приложение}
\phantomsection\addcontentsline{toc}{chapter}{ПРИЛОЖЕНИЕ}
%\section*{\centering\label{code:appendix}Текст программы}

\begin{center}
\label{code:appendix}Текст программы
\end{center}
\begin{code}
код***
\end{code}

\end{document}

